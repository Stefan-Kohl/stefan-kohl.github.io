\documentclass[12pt]{article}

\usepackage{german}
\usepackage[latin1]{inputenc}
\usepackage{theorem}
\usepackage{latexsym}
\usepackage{amsfonts}
\usepackage{amssymb}
\usepackage{amscd}
\usepackage{amsbsy}
\usepackage{amstext}
\usepackage{amsxtra}

\oddsidemargin   0.0cm
\evensidemargin  0.0cm
\topmargin      -1.0cm
\textwidth      16.0cm
\textheight     25.0cm

\pagestyle{empty}

\makeatletter 
\newcommand{\qedChar}{\mbox{$\square$}}
\newcommand{\qed}{{\unskip\nobreak\hfil\penalty50\hskip .001pt \mbox{}
          \nobreak\hfil\qedChar 
           \parfillskip=0pt\finalhyphendemerits=0\medbreak}\rm}
\makeatother 

\begin{document}

\noindent \textbf{Theorem}
\emph{
  Let \(n = p_1 p_2 p_3\) be a Carmichael number with 3 prime factors.
  Without loss of generality, assume that \(p_2 < p_3\).
  Then the following hold:
  \begin{enumerate}
    \item \(p_2 < 2p_1^2\).
    \item \(p_3 < 2p_1^3\).
    \item \(n < 4p_1^6\).
  \end{enumerate}}
\noindent \textbf{Proof:}
  It is an elementary property of Carmichael numbers \(n = p_1 \cdot \; \dots \; \cdot p_k\) that
  \(\forall i \in \{1, \dots, k\} \; (p_i - 1) | (n - 1)\). From this, in case \(k = 3\)
  we conclude
  \begin{itemize}
    \item \((p_3 - 1)|(p_1 p_2 - 1) \Rightarrow p_1 p_2 - 1 = a(p_3 - 1)\) 
          for some \(a \in \mathbb{N}\), and
    \item \((p_2 - 1)|(p_1 p_3 - 1) \Rightarrow p_1 p_3 - 1 = b(p_2 - 1)\) 
          for some \(b \in \mathbb{N}\).
  \end{itemize}
  The assumption \(p_2 < p_3\) implies \(a < b\), and further
  \begin{itemize}
    \item \(a \geq 2\) since \(p_1 p_2 - 1 = 1(p_3 - 1)\) contradicts the primality of~\(p_3\), and
    \item \(b \geq 2\) since \(p_1 p_3 - 1 = 1(p_2 - 1)\) contradicts the primality of~\(p_2\).
  \end{itemize}
  We also get
  \begin{itemize}
    \item \(p_1 p_2 - 1 = a(p_3 - 1) = a p_3 - a \Longrightarrow p_3 = (p_1 p_2 + a - 1)/a\), and
    \item \(p_1 p_3 - 1 = b(p_2 - 1) = b p_2 - b \Longrightarrow p_2 = (p_1 p_3 + b - 1)/b\).
  \end{itemize}
  Inserting the former equation into the latter yields
  \[
    p_2 \ = \ \frac{p_1 (p_1 p_2 + a - 1)/a + b - 1}{b}
        \ = \ \frac{p_1^2 p_2 + ab + (p_1 - 1)a - p_1}{ab}.
  \]
  This implies that
  \[
    (p_1^2 -ab)p_2 + ab + (p_1 - 1)a - p_1 \ = \ 0,
  \]
  from which we get
  \[
    p_2 \ = \ \frac{ab + (p_1 - 1)a - p_1}{ab - p_1^2} 
        \ < \ \frac{ab}{ab - p_1^2} \ + \ \frac{(p_1 - 1)a}{ab - p_1^2}.
  \]
  We have \(ab \neq p_1^2\), since assuming the contrary yields \(a = b = p_1
  \Rightarrow p_1 p_2 - 1 = p_1 (p_3 - 1) \Rightarrow p_3 = p_2 + 1 - 1/p_1 \notin \mathbb{N}\),
  which is not possible. Further it is \(ab > p_1^2\) since \(p_2 > 0\) and
  \(ab + (p_1 - 1)a > 0\). This yields
  \[
    \frac{ab}{ab - p_1^2} \ \leq \ \frac{p_1^2 + 1}{(p_1^2 + 1) - p_1^2} \ = \ p_1^2 + 1.
  \]
  By the assumption \(p_2 < p_3\) we have \(p_1(p_3 - 1) > p_1 p_2 - 1 = a(p_3 - 1)\),
  hence \(a < p_1\). We conclude that
  \[
    \frac{(p_1 - 1)a}{ab - p_1^2} \ < \ \frac{(p_1 - 1)p_1}{ab - p_1^2} \ \leq \ p_1^2 - p_1.
  \]
  This yields \(p_2 < p_1^2 + 1 + p_1^2 - p_1 < 2p_1^2\), as claimed.
  From \(p_1 p_2 - 1 = a(p_3 - 1)\) we get \(p_3 < p_1 p_2 < 2p_1^3\),
  and finally \(n = p_1 p_2 p_3 < p_1 \cdot 2p_1^2 \cdot 2p_1^3 = 4p_1^6\). \qed

\end{document}