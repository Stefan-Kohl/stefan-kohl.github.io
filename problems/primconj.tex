%%%%%%%%%%%%%%%%%%%%%%%%%%%%%%%%%%%%%%%%%%%%%%%%%%%%%%%%%%%%%%%%%%%%%%%%%%%%%%%%%%%%%%%%%%%%%%%%%%%%
%%
%%  primconj.tex                                                                         Stefan Kohl
%%
%%%%%%%%%%%%%%%%%%%%%%%%%%%%%%%%%%%%%%%%%%%%%%%%%%%%%%%%%%%%%%%%%%%%%%%%%%%%%%%%%%%%%%%%%%%%%%%%%%%%

\documentclass[12pt]{article}

\usepackage{german}
\usepackage[latin1]{inputenc}
\usepackage{theorem}
\usepackage{latexsym}
\usepackage{amsfonts}
\usepackage{amssymb}
\usepackage{amscd}
\usepackage{amsbsy}
\usepackage{amstext}
\usepackage{amsxtra}

\oddsidemargin  -0.5cm
\evensidemargin -0.5cm
\textwidth      16.0cm
\textheight     25.0cm

\pagestyle{empty}

\include{macros}

\theoremstyle{change}
\theorembodyfont{\rmfamily}
\newtheorem{Definition}{Definition}
\newtheorem{Conjecture}[Definition]{Conjecture}
\newtheorem{Question}[Definition]{Question}

\begin{document}

\title{Prime Factorization - \(R\) vs. \(R[x]\)}
\author{Stefan Kohl} \date{}
\maketitle

  \begin{Definition}
    Let \(R\) be an euclidean domain all of whose residue class rings are finite.
    We call a function \(f: R \rightarrow R\) \emph{analytic} if it satisfies the condition
    \[
      \forall x,y \in R \ \ f(x) \equiv f(x + y) \!\! \mod \ y.
    \]
    We call the function~\(f\) \emph{reducible} if it can be written as (pointwise) product of
    two other analytic functions \(f_1, f_2\) over~\(R\) whose images do not entirely consist
    of units, and \emph{irreducible} if not.
    We say that a product \(f = \prod_{i=1}^k f_i\) of irreducible analytic functions \(f_i\) 
    over~\(R\) fulfills the \emph{limited factoring condition} if and only if there is an
    \(x \in R\) such that \(f(x)\) has at most \(k\) prime factors.
    We say that \(f\) has \emph{polynomial growth} if for all sequences \(x_1, x_2, \dots\) of
    elements of~\(R\) satisfying \(\lim_{n \rightarrow \infty} |R/x_nR| = \infty\) we have
    \[
      \limsup_{n \rightarrow \infty} \frac{|R/f(x_n)R|}{|R/x_nR|!} \ = \ 0.
    \]
    We call \(R\) a \emph{limited factoring domain} if all analytic functions over~\(R\) of
    polynomial growth fulfill the limited factoring condition.
  \end{Definition}

  \begin{Conjecture}
    The ring \(\ZZ\) is a limited factoring domain.
  \end{Conjecture}

  \begin{Question}
    How to characterize limited factoring domains in general?
  \end{Question}

\end{document}

%%%%%%%%%%%%%%%%%%%%%%%%%%%%%%%%%%%%%%%%%%%%%%%%%%%%%%%%%%%%%%%%%%%%%%%%%%%%%%%%%%%%%%%%%%%%%%%%%%%%
